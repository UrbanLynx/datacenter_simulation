\documentclass[conference]{IEEEtran}
\usepackage{blindtext, graphicx}


\hyphenation{op-tical net-works semi-conduc-tor}

\title{Coflow Scheduling in data center networks}


\begin{document}

\author{ Stas Mushits, Amit Borase, Ruby Pai, Sreejith Unnikrishnan, Ritvik Jaiswal }

\maketitle


\begin{abstract}
%\boldmath
\blindtext[1]
\end{abstract}

\begin{IEEEkeywords}
Coflow scheduling, Data center networks
\end{IEEEkeywords}

\section{Introduction}


\section{Related work}

Our primary work was to study the coflow scheduler work in detail. Network scheduling mostly ignores application level requirements, especially in intensive data parallel applications. At the application layer flow completion time would be the most important metric. Coflow abstraction represents collection of related flows in data network, and improving the coflow completion time is equivalent to improving the application performance. Improving application performance by minimizing coflow completion time (CCT), and ensuring that individual flows meet deadlines is a NP hard problem. Varys is a coflow scheduling algorithm that uses heuristic approach to minimize coflow completion time of the system. The algorithm uses Smallest-Effective-Bottleneck-First (SEBF) heuristic, that schedules coflow based on bottleneck flow`s completion time. Minimum-Allocation-for-Desired-Duration (MADD) algorithm is used to allocate rate for individual flows. MADD slows down all the flows in a coflow to match the completion time of the flow that will take the longest to finish. Working together, these algorithm effectively implements a heuristic that reduces the coflow completion time. Varys coflow scheduler works based on the aformentioned principles. Varys scheduler is designed to be clairvoyant scheduler, meaning it knows the flows and flow structure in advance. However the authors proposed a new algorithm called Aalo, which is non clairvoyant, which in turn translates to a more pragmatic approach in coflow scheudling. Aalo uses multi-level queue structure with a threshold value to effectively reduce flow completion time. The aalo scheduler uses a predefined threshold \(X\) for highest priority queue, and subsequent lower priority queues each multiplied by a factor of \(E^N\). All coflows enter the higher priority queue first. When the flow crosses the threshold it gets de prioritized into lower level queues. This ensures that shorter coflows will get effective bandwidth and is not greatly affected by larger flows. Within a queue, scheduler follows a First In First Out (FIFO) principle.

One of the preliminary task was to create a suitable environment to simulate and run the network. We did a survey of possible network simulators carefully looking for the features it provided and its suitability for our objective. For the baseline network simulation we decided to go for mininet\cite{Mininet}. There are other options like NS2 simulator etc. Support for python programming coupled with integration support with many VMs made our choice of picking up mininet easier. Moreover there is a huge community involvement and documentation for the tool.

The second decision was to decide on a network topology over which we should be running the simulation. Our aim was to ensure that rather than sticking to naive network topologies, we should rather read up on production quality network topologies, that is actually useful in real world scenario. Keeping that in mind and considering the pragmatic implications, we used fat-tree based network topology. Fat tree topology ensures that the end hosts receive full bisection bandwidth and there would be multiple path from a host to core server, ensuring higher bandwidth and reliability. Most of the fat tree implementation requires some method of hashing, usually ECMP to ensure that the multiple paths are effectively utilized. Our search for an SDN controller with ECMP support did not bear the fruit. The next best option was to find a controller that learns MAC addresses for efficient switching. Openflow Floodlight\cite{Floodlight} controller was selected for this purpose. Firstly floodlight supports learning based data forwarding and secondly provides Java based API. We were able to successfully create and deploy a fat tree topology with parameter K = 4, ensuring that we have 16 hosts connected to each other.

We ran the network in our allotted UCSD cluster machines. However we came across a problem when we tried to assign individual IP addresses to each mininet hosts. Mininet host will take IP`s based on the host name, which while running on a single VM is the same. To resolve this, we had to look for other means. After quite a bit of research, we decided to use Docker\cite{Docker} containers for deploying mininet hosts. Docker containers are essentially virtual machines without hypervisor. This essentially means that we were able to create Docker based software containers for individual mininet hosts and was deployed on our cluster virtual machines. Docket provide APIs that enables us to monitor, trigger and execute process in individual containers from the host machine. We used python based API for our interaction with docker containers.

Next step was to create software architecture that is efficient and flexible enough to create the simulation framework. Since the Varys and Aalo scheduler code runs on Scala, we decided to use Java as our main language. Given the legacy of socket programming and multi-threaded capability of Java, our choice proved effective in the long run. Our control traffic was handled in-band, with a master node. The system takes a hosts file that contains the list of hosts and their IP addresses as its input and a task file descibing the frame size, mapper reducer configuration etc. We also provide a simulation file in JSON format which the program uses to identify the Master URL, the URL where the Varys or Aalo scheduler listens to for scheduling the tasks. The Program then runs the simulation in accordance with the tasks file and produce relevant log files as the output. We created a small program in MATLAB, that would analyze the log files to generate graphs and results out of it.

\section{Design}

\section{Evaluation}

\section{Conclusion}

\section*{Acknowledgment}
We would like to thank our instructor Prof. George Porter for his support through out the project and his valuable insights over time. We also would like to thank our TA Yashar for helping us many a times. 

\begin{thebibliography}{10}

\bibitem{Mininet}
Mininet - rapid prototyping for SDN
\textit{https://github.com/mininet/mininet}

\bibitem{Floodlight}
Flood light SDN controller
\textit{https://github.com/floodlight/floodlight}

\bibitem{Docker}
Docker
\textit{https://www.docker.com/}

\end{thebibliography}


\begin{IEEEbiography}[{\includegraphics[width=1in,height=1.25in,clip,keepaspectratio]{picture}}]{John Doe}
\blindtext
\end{IEEEbiography}

\end{document}


