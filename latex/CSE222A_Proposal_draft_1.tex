\documentclass{article}

%\usepackage{amsmath}
%\usepackage{graphicx}
%\usepackage{hyperref}
%\usepackage{listings}
%\lstset{
%numbers=left
%}
\usepackage{url}

\begin{document}

\title{CSE 222A Project Proposal \\
		Coflow Scheduling \\
		\large Group 6}
\author{Ruby Pai \and Amit Borase \and Sreejith Unnikrishnan \and Stas Mushits \and Ritvik Jaiswal}
\date{\today}
\maketitle

\section{Introduction} \label{introduction}

Data parallel cluster applications running on fast CPUs are limited by relatively slower network connections. The coflow abstraction for such applications was recently proposed in \cite{chowdhury2012coflow}, which argued that identifying the individual parallel data flows associated with an application as a "coflow" and applying application-aware scheduling would improve application performance. Subsequently, Varys, a coflow scheduler was published \cite{chowdhury2014varys}, followed by Aalo, which improves on Varys by BRIEF DESCRIPTION HERE \cite{chowdhury2015aalo}. Varys is based on heuristics that draw on insights about characteristics of an optimal coflow scheduler that would allow scheduling for minimal coflow completion time (CCT) or to meet a deadline.

The Varys scheduler was shown to improve CCTs by up to 3.16 times on average, and to increase the number of coflows meeting deadlines by a factor of 2 \cite{chowdhury2014varys}. This is promising not only for application performance, but has implications for issues such as data center energy efficiency, as most of the power wasted in data centers is due to CPUs idling while waiting for the network. However, a significant barrier to usage of coflow schedulers like Varys or Aalo is that they require coflows to be explicitly identified by the programmer to the scheduler via an API. This poses problems in terms of legacy code, programmer burden, and finally, in cases in which the coflows that exist in an application may not be obvious or known. 

Therefore, in this project, we propose to examine coflow scheduling in two parts. First, we will replicate the results of \cite{chowdhury2014varys} using the same trace, which is available on \cite{website:coflow-repo} and extend the experiment to traces for different data parallel applications. Second, taking Varys as the established state-of-the-art in coflow scheduling (as the code for Aalo does not appear to be available yet) we will examine the impact of incomplete coflow identification on the scheduler. Examining the robustness of the current state-of-the-art scheduler to incomplete coflow labeling will inform whether automated algorithms for identifying coflows, using for example machine learning techniques, which cannot be expected to be 100\% accurate, are viable with current coflow scheduling techniques.

\section{Approach}

As mentioned Section \ref{introduction}, the proposed project consists of two parts. The first part is to validate results from published literature, extending the results to new workloads. The second part consists of taking the first step in considering how the published result (in this case, the Varys scheduler) could work with automated coflow identification, which we argue is important for the reasons stated in \ref{introduction}.

\section{Related Work}

So far, we have focused on the seminal work of Chowdhury et al on coflows and coflow scheduling \cite{chowdhury2012coflow, chowdhury2014varys, chowdhury2015aalo}. Though Aalo is the more recent result and the authors argue its overall merits over Varys, the code does not appear to be publically available (we have not contacted the authors). We thus focus on Varys, with a experimental setup that should be easily exendtible to Aalo, as both schedulers are implemented in Scala and expected to work with the coflow simulation code in \cite{website:coflow-repo}. Other work on coflow scheduling includes \cite{zhao2015rapier}, which integrates routing for a performance improvement, as well as other works cited in Section 9 of the August 2015 Aalo paper \cite{chowdhury2015aalo}. Since our primary interest is ultimately in examining the performance of coflow scheduling with less than 100\% labeling of coflows, we chose to use Varys, which is authored by the inventors of the coflow concept and for which the code is available. However, since Varys (AND Aalo) are based on heuristics, it is not currently clear to us what impact this has on the result for all possible coflow schedulers.

We are not currently aware of other work that performs validation of the Varys result and examines the performance of Varys with unidentified coflows, but plan further reading on coflow identification to continue searching for such. 

Additional literature search and reading we need to do are on the coflow concept, coflow scheduling (for example, \cite{chowdhury2014varys} references non-preemptive coflow schedulers, with claims that Varys outperforms by a factor of 5), and coflow identification.

\section{Plan}

First, we will set up a cluster simulation environment. Class resources have made 10 virtual machines available for our use. We plan to use Mininet \cite{website:mininet} (BRIEF DISCUSSION OF MININET) to create a virtual cluster environment wherein each virtual machine contains 10 Mininet hosts, for a total of 100 hosts which will all be connected by Mininet switches as show in figure: INCLUDE FIGURE. Data rate must be selected appropriately to ensure that virtualization does not impact our experiment results, and testing performed to verify the correctness of our setup.

The scheduler performance metrics were coflow completion time (CCT) and number of coflows meeting their deadline over a per-flow scheduler. (It is not clear at a glance how stringent these deadlines were set.) We need think carefully about the evaluation of these metrics in our experimental setup.

--When Done?

\section{Schedule}

By each of the given milestones, we plan to complete the following tasks.

1. Milestones 11/5 (2 weeks)

2. Milestone 11/24 (2.5 weeks)

3. Deliverable 12/8 Poster and Final Report (2 weeks)

%\begin{thebibliography}{9}
%
%\bibitem{varys2014}
%Chowdhury et al
%Varys scheduler paper title.
%
%\end{thebibliography}

\bibliographystyle{plain}
\bibliography{coflow}

\end{document}
 
